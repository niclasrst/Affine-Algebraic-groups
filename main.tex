\documentclass[a4paper, 11pt]{scrartcl}

\addtokomafont{title}{\rmfamily\scshape}
\addtokomafont{section}{\S\;\rmfamily}

% font settings and encodings
\usepackage[T1]{fontenc}
\usepackage[utf8]{inputenc}
\usepackage[ngerman]{babel}

% margins and style
\usepackage{setspace}
\usepackage{enumerate}
\usepackage[top=2cm, bottom=2cm, left=2.5cm, right=2.5cm]{geometry}
\geometry{a4paper}

% math packages
\usepackage{amsmath}
\usepackage{amsthm}
\usepackage{amsfonts}
\usepackage{amssymb}

% math diagrams
\usepackage{tikz-cd}
\usetikzlibrary{babel}

% referencing
\usepackage[pdftex,bookmarks,colorlinks,breaklinks]{hyperref}
\hypersetup{linkcolor=black,citecolor=black,filecolor=black,urlcolor=black}
\usepackage[nameinlink,german]{cleveref}
\Crefname{proposition}{Proposition}{Propositionen}
\usepackage{url}

% shortcuts
\newcommand{\N}{\mathbb{N}}
\newcommand{\Z}{\mathbb{Z}}
\newcommand{\R}{\mathbb{R}}
\newcommand{\C}{\mathbb{C}}
\newcommand{\K}{\mathbb{K}}
\newcommand{\ringquot}[2]{{\raisebox{.2em}{\(#1\)}/\raisebox{-.2em}{\(#2\)}}}
\newcommand{\brackets}[1]{\left\langle #1 \right\rangle}
\newcommand{\Hom}{\text{Hom}}
\newcommand{\Nat}{\text{Nat}}
\newcommand{\id}{\text{id}}
\newcommand{\cod}{\text{cod}}
\newcommand{\dom}{\text{dom}}
\newcommand{\Ob}{\text{Ob}}
\newcommand{\Ar}{\text{Ar}}
\newcommand{\F}{\mathcal{F}}
\newcommand{\CC}{\mathcal{C}}
\newcommand{\D}{\mathcal{D}}
\newcommand{\polynomials}{k[X_1, \ldots, X_n]}

% math styles
\newtheoremstyle{basicstyle}{1em}{1em}{\normalfont}{}{\bfseries}{}{ }{}

% math environments
\theoremstyle{basicstyle}
\newtheorem{definition}{Definition}[section]
\newtheorem{bemerkung}[definition]{Bemerkung}
\newtheorem{beispiel}[definition]{Beispiel}
\newtheorem{lemma}[definition]{Lemma}
\newtheorem{korollar}[definition]{Korollar}
\newtheorem{satz}[definition]{Satz}
\newtheorem{proposition}[definition]{Proposition}
\newtheorem{erinnerung}[definition]{Erinnerung}

% turn off page numbering
\pagenumbering{gobble}

\title{Algebraische Gruppen}
\author{Niclas Rist}
\date{Algebra Seminar SS24}

\begin{document}
    \maketitle
    \onehalfspacing

    % ****************************************
    \section{Kategorien, Funktoren \& natürliche Transformationen}
    % ****************************************

    \begin{definition}
        Eine \emph{Kategorie} \(\mathcal{C}\) besteht aus einer Klasse von \emph{Objekten} \(\Ob(\mathcal{C})\) und einer Klasse von \emph{Morphismen} \(\Hom_{\mathcal{C}}(A, B)\) für je zwei Objekte \(A, B \in \Ob(\mathcal{C})\).
        Morphismen
        \[(A \overset{f}{\longrightarrow} B) := (f: A \longrightarrow B) \in \Hom_{\mathcal{C}}(A, B)\]
        werden auch \emph{Pfeile} genannt.
        Sie unterliegen den folgenden Axiomen:
        \begin{enumerate}
            \item Morphismen können \emph{verknüpft} werden, für Morphismen \(f: A \to B, \; g: B \to C\) erhalten wir \(g \circ f: A \to C\), und die Komposition ist assoziativ, d.h. für \(A \overset{f}{\longrightarrow} B \overset{g}{\longrightarrow} C \overset{h}{\longrightarrow} D\) gilt
                \[h \circ (g \circ f) = (h \circ g) \circ f.\]
            \item Für jedes \(A \in \Ob(\mathcal{C})\) gibt es einen \emph{Identitätsmorphismus} \(\id_A \in \Hom_{\mathcal{C}}(A, A)\), sodass für alle \(f: A \to X\) und \(g: Y \to A\), mit \(X, Y \in \Ob(\mathcal{C})\) gilt
                \[f \circ \id_A = f \qquad\text{und}\qquad \id_A \circ g = g.\]
        \end{enumerate}
        Ein Morphismus \(A \overset{f}{\longrightarrow} B\) in einer Kategorie \(\mathcal{C}\) ist ein \emph{Isomorphismus}, wenn es einen Morphismus \(B \overset{g}{\longrightarrow} A\) gibt, genannt die \emph{Inverse} von \(f\), sodass \(g \circ f = \id_A\) und \(f \circ g = \id_B\) gilt.
    \end{definition}

    \begin{bemerkung}
        Bemerke, dass \(\Ob(\mathcal{C})\) und \(\Hom_{\mathcal{C}}\) Klassen und keine Mengen sind.
        Dadurch werden Mengentheoretische Paradoxa umgangen (z.B. die Menge aller Mengen), diese sind für uns aber nicht weiter von Belang.
        Man kann noch anmerken, dass viele Kategorien jedoch \emph{lokal klein} sind, d.h. \(\Hom_{\mathcal{C}}(A, B)\) ist tatsächlich eine Menge für alle \(A, B \in \Ob(\mathcal{C})\).
    \end{bemerkung}

    \begin{beispiel}\leavevmode
        Einige Beispiele von Kategorien sind die Folgenden.
        \begin{figure*}[ht!]
            \centering
            \setlength{\tabcolsep}{10pt}
            \renewcommand{\arraystretch}{1.5}
            \begin{tabular}{|c|c|c|c|}
                \hline
                \textbf{Kategorie} & \textbf{Objekte}   & \textbf{Morphismen}   & \textbf{Isomorphismen} \\ \hline
                {\textbf{Set}}          & Mengen                 & Abbildungen               & Bijektive Abb.             \\ \hline
                {\textbf{Grp}}          & Gruppen                & Gruppenhom.               & Gruppeniso.                \\ \hline
                {\textbf{Top}}          & Topologische Räume     & stetige Abbildungen       & Homöomorphismen            \\ \hline
                {\(\textbf{Vec}_{k}\)}  & Vektorräume über \(k\) & \(k\)-lineare Abbildungen & \(k\)-Isomorphismen        \\ \hline
                {\textbf{\(k\)-Alg}}    & Algebren über \(k\)    & \(k\)-Alg. Hom.           & \(k\)-Isomorphismen        \\ \hline
            \end{tabular}
        \end{figure*}
    \end{beispiel}

    \begin{bemerkung}
        Falls \(f: A \to B\) ein Isomorphismus ist und \(g_1, g_2\) zwei Inverse von \(f\), dann gilt
        \[g_2 = g_2 \circ \id_B = g_2 \circ (f \circ g_1) = (g_2 \circ f) \circ g_1 = \id_A \circ g_1 = g_1.\]
        Inverse Morphismen sind also eindeutig, sofern sie existieren.
    \end{bemerkung}

    \begin{erinnerung}
        Sei \(R\) ein kommutativer Ring mit \(1\).
        \begin{enumerate}
            \item Eine \emph{\(R\)-Algebra} ist ein (nicht unbedingt kommutativer) Ring \(A\) mit Einselement, der auch ein \(R\)-Modul ist, sodass die Multiplikation in \(A\) bilinear ist und
                \[\lambda x \cdot y = x \cdot \lambda y = \lambda \cdot (xy)\]
                für alle \(x,y \in A\) und alle \(\lambda \in R\) gilt.

            \item Das \emph{Zentrum} einer \(R\)-Algebra \(A\) ist die Unteralgebra
                \[Z(A) := \{a \in A \;\vert\; ax = xa \; \text{für alle} \; x \in A\}.\]

            \item Der sogenannte \emph{Strukturhomomorphismus} 
                \[\varphi: R \longrightarrow A, \qquad \lambda \mapsto \lambda \cdot 1_A\]
                ist ein Ringhomomorphismus von \(R\) in das Zentrum \(Z(A)\) von \(A\).

            \item Ein Morphismus zwischen \(k\)-Algebren \(A\) und \(B\) ist ein Ringhomomorphismus
                \[\xi: A \longrightarrow B, \qquad\text{mit}\qquad \xi \circ \varphi_A = \varphi_B.\]
        \end{enumerate}
    \end{erinnerung}

    \begin{definition}
        Sei \(\mathcal{C}\) eine Kategorie.
        Die \emph{duale Kategorie} \(\mathcal{C}^{\text{op}}\) (op von engl. `opposite') ist die Kategorie mit umgekehrten Pfeilen.
        Ausgedrückt mit den Klassen der Objekte und Morphismen,
        \[\Ob(\mathcal{C}^{\text{op}}) = \Ob(\mathcal{C}), \quad \Hom_{\mathcal{C}^{\text{op}}}(X, Y) = \Hom_{\mathcal{C}}(Y, X).\]
    \end{definition}

    \begin{beispiel}
        Man kann sich überlegen, wie die duale Kategorie der \(k\)-Algebren aussieht.
        Dies führt auf die Kategorie der \(k\)-Co-Algebren, welche wir mit noch mehr Struktur als Hopf-Algebren (welche gerade gleichzeitig Co-Algebren und Algebren, sg. Bi-Algebren sind, zusätzlich ausgestattet mit einer Antipode) im Zusammenhang mit Koordinatenringen affiner algebraischer Gruppen wiederentdecken werden.
    \end{beispiel}

    \begin{definition}
        % Ein \emph{(kovarianter) Funktor} \(\mathcal{F}:\mathcal{C} \to \mathcal{D}\) zwischen Kategorien \(\mathcal{C}\) und \(\mathcal{D}\) ist eine Abbildung auf der Klasse von Objekten und der Klasse der Pfeile.
        Ein \emph{(kovarianter) Funktor} \(\mathcal{F}: \mathcal{C} \to \mathcal{D}\) zwischen Kategorien \(\mathcal{C}\) und \(\mathcal{D}\) ist eine Zuweisung von Objekten, als auch von Pfeilen zwischen \(\mathcal{C}\) und \(\mathcal{D}\).
        Sie weißt jedem \(A \in \Ob(\mathcal{C})\) ein Objekt \(\mathcal{F}(A) \in \mathcal{D}\) und jedem Pfeil \((f: A \to B) \in \Hom_\mathcal{C}(A, B)\) zwischen Objekten \(A, B \in \Ob(\mathcal{C})\) einen Pfeil \(\mathcal{F}(f) \in \Hom_{\mathcal{D}}(\mathcal{F}(A), \mathcal{F}(B))\) in \(\mathcal{D}\) zu, sodass die folgenden Axiome erfüllt sind
        \begin{enumerate}
            \item \(\mathcal{F}(f \circ g) = \mathcal{F}(f) \circ \mathcal{F}(g)\) für alle \(f \in \Hom_{\mathcal{C}}(A, B), \; g \in \Hom_{\mathcal{D}}(B, C)\).
            \item \(\mathcal{F}(\id_A) = \id_{\mathcal{F}(A)}\) für alle \(A \in \Ob(\mathcal{C})\).
        \end{enumerate}
        % \begin{figure*}[ht!]
        %     \centering
        %     \begin{tikzcd}
        %         X \arrow[r, "\mathcal{F}"]\arrow[d, "f"] & \mathcal{F}(X) \arrow[d, "\mathcal{F}(f)"] \\
        %         Y \arrow[r]          & \mathcal{F}(Y)
        %     \end{tikzcd}
        % \end{figure*}
        Ein \emph{kontravarianter Funktor} \(\mathcal{T}: \mathcal{C} \to \mathcal{D}\) ist ein kovarianter Funktor von \(\mathcal{C}^{\text{op}}\) nach \(\mathcal{D}\).
        Also erfüllt \(\mathcal{T}\) die obigen Eigenschaften, außer dass nun gilt:
        \[\mathcal{T}(f) \in \Hom_{\mathcal{D}}(\mathcal{T}(B), \mathcal{T}(A)) \qquad\text{und}\qquad \mathcal{T}(f \circ g) = \mathcal{T}(g) \circ \mathcal{T}(f).\]
    \end{definition}

    \begin{beispiel}
        \begin{enumerate}
            \item Der \emph{Vergiss-Funktor} \(\text{forget}: \textbf{Grp} \to \textbf{Set}\), \((G, +) \mapsto G\) `vergisst' einen Teil der Struktur den die Objekte einer Kategorie besitzen.
            \item Der zur Abelisierung gehörende Funktor \((\;\cdot\;)_{\text{ab}}: \textbf{Grp} \to \textbf{Ab}, \; G \mapsto \ringquot{G}{[G,G]}\), wobei \([G,G]\) die kommutator Untergruppe erzeugt von \(g_1g_2g_1^{-1}g_2^{-1}\) bezeichnet.
                Auf Morphismen ist dieser gegeben durch
                \[(f: G \to H) \mapsto (G_{\text{ab}} \to H_{\text{ab}}, \; [g] \mapsto [f(g)]).\]
            \item Der kontravariante \emph{Dualraum-Funktor}
                \[\Hom_{\textbf{Vec}_k}(\;\cdot\;, k): \textbf{Vec}_k \to \textbf{Vec}_k, \qquad V \mapsto V^\ast = \Hom_k(V,k).\]
                Auf Morphismen wirkt dieser durch
                \[(f: V \to W) \mapsto (f^\ast : W^\ast \to V^\ast, \; \varphi \mapsto \varphi \circ f).\]
        \end{enumerate}
    \end{beispiel}

    Wir verlassen die Funktoren mit folgendem Hilfsresultat.

    \begin{lemma}\label{lem:3}
        Seien \(\CC, \D\) Kategorien und \(\F: \CC \to \D\) ein \emph{volltreuer}\footnote[1]{heißt bijektiv auf den Morphismen} Funktor.
        Dann gilt:
        \[\F(A) \cong_\D \F(B) \quad\iff\quad A \cong_\CC B.\]
    \end{lemma}
    \begin{proof}
        Es gelte \(\F(A) \cong_\D \F(B)\), dann gibt es Morphismen
                \[\F(X) \xrightarrow{\quad\gamma\quad} \F(Y) \qquad\text{und}\qquad \F(Y) \xrightarrow{\quad\delta\quad} \F(X)\]
                sodass \(\gamma \circ \delta = \id_{\F(X)}\) und \(\delta \circ \gamma = \id_{\F(Y)}\).
                Da \(\F\) voll ist, gibt es 
                \[X \xrightarrow{\quad\alpha\quad} Y \qquad\text{und}\qquad Y \xrightarrow{\quad\beta\quad} X\]
                mit \(\gamma = \F(\alpha)\) und \(\delta = \F(\beta)\).
                Dann sind \(\alpha \circ \beta\) und \(\id_X\) zwei Morphismen mit 
                \[\F(\alpha \circ \beta) = \F(\id_X) = \id_{\F(X)}\]
                und da \(\F\) treu ist, folgt \(\alpha \circ \beta = \id_X\).
                Analog gilt \(\beta \circ \alpha = \id_Y\).
    \end{proof}

    \begin{definition}
        Eine \emph{natürliche Transformation} \(\eta : \mathcal{F} \to \mathcal{T}\) ist eine Abbildung zwischen zwei Funktoren \(\mathcal{F}, \mathcal{T}: \mathcal{C} \rightrightarrows \mathcal{D}\).
        Sie weißt jedem Objekt \(A \in \Ob(\mathcal{C})\) einen Morphismus
        \[(\eta_A : \mathcal{F}(A) \to \mathcal{T}(A)) \in \Hom_\mathcal{D}(\mathcal{F}(A), \mathcal{T}(A))\] in \(\mathcal{D}\) zu, sodass das folgende Diagramm für alle \((f: A \to B) \in \Hom_{\mathcal{C}}(A,B)\) kommutiert.
        \begin{figure*}[ht!]
            \centering
            \begin{tikzcd}
                A \arrow[d, "f"] \\
                B
            \end{tikzcd}
            \(\qquad\rightsquigarrow\qquad\)
            \begin{tikzcd}
                \mathcal{F}(A) \arrow[r, "\eta_A"] \arrow[d, "\mathcal{F}(f)"'] & \mathcal{T}(A) \arrow[d, "\mathcal{T}(f)"] \\
                \mathcal{F}(B) \arrow[r, "\eta_B"'] & \mathcal{T}(B)
            \end{tikzcd}
        \end{figure*} \\
        Der Morphismus \(\eta_A\) wird \emph{Komponente} von \(\eta\) in \(A \in \Ob(\mathcal{C})\) genannt.
        Eine natürliche Transformation heißt \emph{natürlicher Isomorphismus}, wenn alle Komponenten \(\eta_A\) Isomorphismen sind.
    \end{definition}

    \begin{bemerkung}\label{bem:1}
        Natürliche Transformationen werden auch als \emph{Morphismen von Funktoren} bezeichnet.
        Dies rührt daher, dass man sogenannte \emph{Funktorkategorien} \(\textbf{Fun}(\CC, \D)\) definieren kann.
        Als Objekte haben diese Funktoren zwischen zwei Kategorien und Morphismen sind gerade die natürlichen Transformationen zwischen Funktoren.
    \end{bemerkung}

    \begin{beispiel}
        Betrachte den Bidualraum-Funktor \((\;\cdot\;)^{\ast\ast}: \textbf{Vec}_k \to \textbf{Vec}_k\), gegeben durch
        \begin{equation*}
            V \to V^{\ast\ast}, \qquad
            (f: V \to W) \mapsto \begin{cases}
                f^{\ast\ast}: V^{\ast\ast} \to W^{\ast\ast}, \\
                (\delta: V^\ast \to k) \mapsto (\varphi \mapsto \delta(\varphi \circ f))
            \end{cases}
        \end{equation*}
        Weiter sei \(\id: \textbf{Vec}_k \to \textbf{Vec}_k\) der Identitätsfunktor.
        Dann ist \(\eta: \id \to (\;\cdot\;)^{\ast\ast}\) gegeben durch
        \begin{equation*}
            \eta_V : V \longrightarrow V^{\ast\ast}, \qquad v \mapsto \begin{cases}
                ev_v: V^\ast \to k \\
                \varphi \mapsto \varphi(v)
            \end{cases}
        \end{equation*}
        eine natürliche Transformation zwischen den beiden Funktoren.
    \end{beispiel}

    \begin{bemerkung}
        Beschränkt man sich in obigem Beispiel auf \underline{endlichdimensionale} Vektorräume, so sind \(V\) und sein Bidualraum \(V^{\ast\ast}\) natürlich isomorph mit \(\eta\).
        Nun ist auch \(V\) isomorph zu seinem Dualraum \(V^\ast\) aber nicht natürlich isomorph!
        Der Isomorphismus \(V \to V^\ast\) hängt von der Wahl einer Basis in \(V\) ab, aber es gibt keine `natürliche' Wahl.
    \end{bemerkung}

    \begin{definition}
        Eine \emph{Äquivalenz von Kategorien} \(\mathcal{C}\) und \(\mathcal{D}\) besteht aus (kovarianten) Funktoren \(\mathcal{F} : \mathcal{C} \to \mathcal{D}\) und \(\mathcal{T} : \mathcal{D} \to \mathcal{C}\) so, dass \(\mathcal{F} \circ \mathcal{T}\) und \(\mathcal{T} \circ \mathcal{F}\) natürlich isomorph zu \(\id_{\mathcal{D}}\), bzw. \(\id_{\mathcal{C}}\) sind.
        Sind die Funktoren \(\mathcal{F}, \mathcal{T}\) kontravariant, so spricht man von einer \emph{Antivalenz} oder \emph{Dualität}.
    \end{definition}

    % \begin{bemerkung}
    %     Sei \(\mathcal{C}\) eine Kategorie.
    %     Ein Objekt \(X \in \Ob(\mathcal{C})\) definiert einen Funktor
    %     \begin{equation*}
    %         h^X: \mathcal{C} \longrightarrow \textbf{Set}, \quad
    %         \begin{cases}
    %             h^X (Y) = \Hom_{\mathcal{C}}(X, Y), \quad Y \in \Ob(\mathcal{C}), \\
    %             h^X (f)(g) = f \circ g, \quad f: Y \to Y', \quad g \in h^X(Y) = \Hom_{\mathcal{C}}(X, Y).
    %         \end{cases}
    %     \end{equation*}
    %     % Ein Morphismus \(\varphi: X' \to X\) induziert eine natürliche Transformation 
    %     % \[\eta_Y: h^X(Y) \to h^{X'}(Y), \qquad f \mapsto f \circ \alpha,\]
    %     % was die Zuordnung \(X \rightsquigarrow h^X\) zu einem kovarianten Funktor macht.
    %     Wir schreiben auch \(h^X := \Hom_{\mathcal{C}}(X, \;\cdot\;)\) und nennen \(h^X\) den \emph{(partiellen) Hom-Funktor}.
    % \end{bemerkung}

    \begin{definition}\label{def:darstellbar}
        Sei \(\mathcal{C}\) eine (lokal kleine) Kategorie, dann ist für jedes Objekt \(A \in \Ob(\mathcal{C})\)
        \[h^A: \mathcal{C} \longrightarrow \textbf{Set},\]
        ein kovarianter Funktor, gegeben auf Objekten durch
        \[X \;\mapsto\; \Hom_{\mathcal{C}}(A, X)\]
        und auf Morphismen durch
        \[\left(X \overset{f}{\longrightarrow} Y\right) \quad\mapsto\quad \begin{pmatrix} \Hom_{\mathcal{C}}(A, X) \longrightarrow \Hom_{\mathcal{C}}(A, Y) \\ g \mapsto g \circ f \end{pmatrix}\]
        Ein Funktor \(\mathcal{F}: \mathcal{C} \to \textbf{Set}\) heißt \emph{darstellbar}, wenn es ein Objekt \(A\) in \(\Ob(\mathcal{C})\) gibt, sodass \(\mathcal{F}\) isomorph zu \(h^A\) ist. %er isomorph zu einem Funktor \(h^X\) für ein \(X \in \Ob(\mathcal{C})\) ist.
        Wir sagen dann, \(A\) stellt \(\mathcal{F}\) dar.
        % Ein Paar \((A, a)\) mit \(a \in \mathcal{F}(A)\) \emph{repräsentiert} \(\mathcal{F}\), wenn die natürliche Transformation gegeben durch
        % \[\eta_a : h^A = \Hom_\mathcal{C}(A, \;\cdot\;) \longrightarrow \mathcal{F}(\;\cdot\;), \quad f \mapsto \mathcal{F}(f)(a)\]
        % natürlicher Isomorphismus ist.
    \end{definition}

    \begin{definition}
        Sei \(\mathcal{F}: \mathcal{C} \to \textbf{Set}\) ein Funktor und \(A \in \Ob(\mathcal{C})\).
        Jede natürliche Transformation \(\eta: h^A \to \mathcal{F}\) definiert ein Element in \(\mathcal{F}(A)\) wie folgt
        \[a_{\eta} := \eta_A(\id_A) \in \mathcal{F}(A).\]
        Umgekehrt definiert jedes Element \(a \in \mathcal{F}(A)\) eine natürliche Transformation \(\eta^a: h^A \to \mathcal{F}\) durch
        \[\eta_X^a: h^A(X) \longrightarrow \mathcal{F}(X), \qquad g \mapsto \mathcal{F}(g)(a).\]
    \end{definition}

    \begin{bemerkung}
        In obiger Definition ist \(g \in \Hom_{\mathcal{C}}(A, X)\).
        Da \(\mathcal{F}\) ein kovarianter Funktor ist, ist \(\mathcal{F}(g) \in \Hom_{\textbf{Set}}(\mathcal{F}(A), \mathcal{F}(X))\) und damit \(\mathcal{F}(g)(a)\) tatsächlich in \(\mathcal{F}(X)\).
        % Die Abbildung \(\eta_X^a\) ist also wohldefiniert.
    \end{bemerkung}

    \begin{lemma}[Yoneda]\label{lem:yoneda}
        Sei \(\mathcal{F}: \mathcal{C} \to \textbf{Set}\) ein Funktor von einer Kategorie \(\mathcal{C}\) in die Kategorie \textbf{Set} und \(A \in \Ob(\mathcal{C})\) ein Objekt mit Hom-Funktor \(h^A : \mathcal{C} \to \textbf{Set}\).
        Dann ist die Abbildung
        \[\theta : \text{Nat}(h^A, \mathcal{F}) \overset{\cong}{\longrightarrow} \mathcal{F}(A), \qquad \eta \mapsto a_{\eta} = \eta_A(\id_A)\]
        % zwischen den natürlichen Transformationen von \(h^X\) nach \(\mathcal{F}\), und den Elementen in \(\mathcal{F}(X)\).
        eine Bijektion; ihre Inverse ist gegeben durch die Abbildung
        \[\xi : \mathcal{F}(A) \longrightarrow \text{Nat}(h^A, \mathcal{F}), \qquad a \mapsto \eta^a\]
    \end{lemma}
    \begin{proof}[Ohne Beweis.]
    \end{proof}

    \begin{korollar}\label{kor:1}
        Sei \(\mathcal{C}\) eine (lokal kleine) Kategorie.
        Für jedes Paar \(A, B \in \Ob(\mathcal{C})\) gilt % es eine Bijektion
        \[\text{Nat}(h^A, h^B) \cong \Hom_{\mathcal{C}}(B, A).\]
    \end{korollar}
    \begin{proof}
        Dies ist ein Spezialfall des Yoneda-Lemma mit \(\mathcal{F} = h^B\).
    \end{proof}

    \begin{bemerkung}\label{bem:2}
        Insbesondere erhalten wir aus \Cref{kor:1}, dass der Funktor
        \[\mathcal{C} \longrightarrow \mathcal{C}^\vee, \qquad A \mapsto h^A,\]
        von der Kategorie \(\mathcal{C}\) in die Funktorkategorie \(\mathcal{C}^\vee = \textbf{Fun}(\mathcal{C}, \textbf{Set})\) kontravariant und volltreu ist. % Morphismen in C^\vee sind genau Nat. Trafos!!
        Durch Einschränkung erhalten wir eine Dualität zwischen der Kategorie \(\CC\) und der (vollen\footnote[1]{eine Unterkategorie \(\mathcal{U}\) (ist das was man sich auch darunter vorstellt) heißt voll, wenn für alle Objektpaare in \(\CC\) gilt \(\Hom_{\C}(A, B) = \Hom_{\mathcal{K}}(A, B)\). Siehe z.B. \url{https://de.wikipedia.org/wiki/Kategorientheorie}.} Unter-) Kategorie der darstellbaren Funktoren \(\CC_{\text{rep}}^\vee\).
        Weiter gilt mit \Cref{lem:3} für \(A, B \in \Ob(\CC)\)
        \[h^A \cong_{\CC^\vee} h^B \quad\iff\quad A \cong_\CC B.\]
    \end{bemerkung}

    \newpage

    % ****************************************
    \section{Affine Varietäten und Funktoren}
    % ****************************************
    
    {\itshape Durchweg ist \(k\) ein Körper und `Ring' heißt kommutativer Ring mit \(1\).}
    
    \begin{definition}
        \begin{enumerate}
            \item Eine \emph{affine Varietät} ist eine Teilmenge des \(k^n\) definiert durch die Nullstellen einer Menge von Polynomen in \(k[X_1, \ldots, X_n]\).
                Für \(S \subseteq k[X_1, \ldots, X_n]\) ist also 
                \[V(S) := \{x \in k^n \;\vert\; f(x) = 0 \; \text{für alle}\; f \in S\}\]
                die affine Varietät \(V(S)\).

            \item Umgekehrt ist für eine Teilmenge \(V \subseteq k^n\) die Menge
                \[I(V) := \{f \in k[X_1, \ldots, X_n] \;\vert\; f(x) = 0 \;\text{für alle}\; x \in V\}\]
                das sogenannte \emph{Verschwindungsideal} zu \(V\).
        \end{enumerate}
    \end{definition}

    Bemerke, dass jede affine Varietät von endlich vielen Polynomen definiert wird.
    Dies folgt aus dem Hilbertschen Basissatz, der besagt, dass jedes Ideal \(I \trianglelefteq k[X_1, \ldots, X_n]\) endlich erzeugt ist.

    \begin{satz}[Hilbertscher Nullstellensatz]
        Sei \(k\) algebraisch abgeschlossener Körper, \(n \in \N\) und \(I \trianglelefteq k[X_1, \ldots, X_n]\) ein Ideal.
        Dann ist \(I(V(I)) = \sqrt{I}\), also \(= I\) gdw. \(I\) ein Radikalideal ist.
    \end{satz}

    \begin{bemerkung}
        Der Hilbertsche Nullstellensatz ist ein erster Ansatz, geometrische Objekte (affine Varietäten) mit algebraischen Objekten (Radikalidealen) zu verknüpfen.
        Weiter kann man jeder affinen Varietät \(V\) ihren Koordinatenring \(k[V] = \ringquot{\polynomials}{I(V)}\) zuweisen um diese Verknüpfung weiter auszubauen.
        Man kann dann zeigen:
    \end{bemerkung}

    \begin{proposition}
        Die Kategorie der affinen Varietäten über algebraisch abgeschlossem Körper \(k\) ist dual (oder antivalent) zur Kategorie der endlich erzeugten, reduzierten \(k\)-Algebren.
    \end{proposition}
    \begin{proof}[Ohne Beweis.]
    \end{proof}

    Wir wollen nun versuchen, uns von der Einschränkung dass \(k\) algebraisch abgeschlossen ist zu lösen.
    Jedoch haben wir dann den Hilbertschen Nullstellensatz nicht mehr zur Verfügung.

    \begin{beispiel}\label{bsp:1}
        Sei \(I = \brackets{X^2 + 1} \trianglelefteq k[X]\) Ideal.
        \begin{enumerate}
            \item Für \(k = \R\) ist \(V(I) = \emptyset\), damit \(I(V(I)) = I(\emptyset) = k[X] \neq I\).
            \item Für \(k = \C\) ist hingegen \(V(I) = \{-i, i\}\) und damit 
                \[I(V(I)) = \brackets{X - i} \cap \brackets{X + i} = \brackets{X^2 + 1} = I.\]
        \end{enumerate}
    \end{beispiel}

    % Die Geometrie wird also nicht mehr genügend Informationen enthalten in dem Sinne, dass Punkte die wir eigentlich sehen sollten fehlen.
    % Wir wollen im Folgenden die geometrischen Objekte über allen Körpererweiterungen und sogar \(k\)-Algebraerweiterungen \emph{gleichzeitig} betrachten.
    % Im Folgenden sehen wir, dass die Sichtweise der Funktoren wie dafür gemacht ist.
    Es fehlen unserer Geometrie also Punkte, die aus algebraischer Sicht vorhanden sein sollten.
    Wir wollen im folgenden daher die geometrischen Objekte über beliebigen Körper- und sogar \(k\)-Algebraerweiterungen betrachten.
    % Dies wollen wir am liebsten simultan tun und wir werden sehen, dass sich die funktorielle Sicht hierfür aufdrängt.
    Dies wollen wir tun, indem wir die algebraischen Informationen abstrakter in Funktoren verschlüsseln.
    Der Vorteil liegt dann darin, dass wir nicht an \emph{eine} Körpererweiterung gebunden sind, sondern alle (in gewisser Weise) \emph{simultan} betrachten können.

    \begin{definition}\label{def:2}
        \begin{enumerate}
            \item Ein \emph{\(k\)-Funktor} \(\mathcal{F}\) ist ein Funktor von der Kategorie der (endlich erzeugten) \(k\)-Algebren in die Kategorie \textbf{Set}, also ein Funktor der Form \(\mathcal{F}: k\textbf{-Alg} \to \textbf{Set}\).
            
            \item Zu einer \(k\)-Algebra \(A \in \Ob(k\textbf{-Alg})\) betrachte den \(k\)-Funktor
                \[h^A : k\textbf{-Alg} \longrightarrow \textbf{Set}, \qquad R \mapsto \Hom_{k\textbf{-Alg}}(A, R).\]
                Ein \(k\)-Funktor \(\mathcal{F}\) heißt \emph{affin}, wenn es eine endlich erzeugte \(k\)-Algebra \(A\) gibt, mit \(h^A \cong \mathcal{F}\).
                Die Algebra \(A\) \emph{repräsentiert} also den \(k\)-Funktor \(\mathcal{F}\) und ist als solche eindeutig bis auf Isomorphismus (cf. \Cref{bem:2}); \(A\) wird auch \emph{Koordinatenring} oder \emph{Koordinatenalgebra} von \(\mathcal{F}\) genannt.
                Wir bezeichnen \(A\) auch mit \(A = k[\mathcal{F}]\).

            \item Ein Morphismus von affinen \(k\)-Funktoren ist eine natürliche Transformation \(\eta: \F \to \mathcal{G}\).
                Diese korrespondiert nach \Cref{kor:1} zu einem eindeutigen \(k\)-Algebrahomomorphismus \[\varphi: k[\mathcal{G}] \to k[\F].\]
        \end{enumerate}
    \end{definition}

    \begin{definition}
        Sei \(k\) ein Körper, \(I \trianglelefteq k[X_1, \ldots, X_n]\) ein Ideal und \(A = \ringquot{k[X_1, \ldots, X_n]}{I}\) Faktorring.
        Für jede \(k\)-Algebra \(R \in \Ob(k\textbf{-Alg})\) definieren wir die Menge
        \[V_R(I) := \{x \in R^n \;\vert\; f(x) = 0 \;\text{für alle}\; f \in I\},\]
        genannt die Menge der \emph{\(R\)-wertigen Punkte} von \(A\).
    \end{definition}

    \begin{bemerkung}
        Sei \(I \trianglelefteq \polynomials\) ein Ideal, definiere den \(k\)-Funktor
        \[V_{(\;\cdot\;)}(I): k\textbf{-Alg} \longrightarrow \textbf{Set}, \qquad R \mapsto V_R(I).\]
        Bezeichne weiter mit \(ev_x\) den Auswertungshomomorphismus, gegeben durch
        \[ev_x : A = \ringquot{\polynomials}{I} \longrightarrow R, \qquad f \mapsto f(x),\]
        dann erhalten wir für jedes \(R \in \Ob(k\textbf{-Alg})\) eine Bijektion
        \[V_R(I) \longrightarrow \Hom_{k\textbf{-Alg}}(A, R), \qquad x \mapsto ev_x.\]
        Wir nennen den Funktor \(V_{(\;\cdot\;)}(I)\) auch den \emph{Punktfunktor} zu \(I\).
        Nach \Cref{def:darstellbar} wird dieser durch die \(k\)-Algebra \(A = \ringquot{k[X_1, \ldots, X_n]}{I}\) dargestellt, da nach obigem gilt: \(V_{(\;\cdot\;)}(I) \cong h^A\).
        % Dies bedeutet gerade, dass der sogenannte \emph{Punktfunktor} \(V_{(\;\cdot\;)}(I): k\textbf{-Alg} \to \textbf{Set}\) isomorph zu dem Funktor \(h^A = \Hom_{k\textbf{-Alg}}(A, \;\cdot\;)\) ist.
        % Nach \Cref{def:darstellbar} wird der Punktfunktor also durch die \(k\)-Algebra \(A\) dargestellt.
    \end{bemerkung}
    \begin{proof}
        \begin{enumerate}
            \item {\itshape Injektivität:}
                Sei \(X_i \in k[X_1, \ldots, X_n]\) die \(i\)-te Koordinatenfunktion und \(x_i\) ihr Bild im Faktorring \(A = \ringquot{k[X_1, \ldots, X_n]}{I}\).
                Für \(v = (v_1, \ldots, v_n) \in V_R(I)\) ist
                \[ev_v(x_i) = x_i(v) = X_i(v) = v_i,\]
                also
                \[v = (ev_v(x_i), \ldots, ev_v(x_n)).\]
                Damit ist \(v\) eindeutig durch \(ev_v\) festgelegt, die Abbildung also injektiv.

            \item {\itshape Surjektivität:} Sei \(\varphi \in \Hom_{k\textbf{-Alg}}(A, R)\) und 
                \[x := (\varphi(x_1), \ldots, \varphi(x_n)) \in R^n.\]
                Betrachte \(\widetilde{\varphi} \in \Hom_{k\textbf{-Alg}}(k[X_1, \ldots, X_n], R)\), gegeben durch die Komposition
                \[k[X_1, \ldots, X_n] \xrightarrow{\;\text{proj.}\;} \ringquot{k[X_1, \ldots, X_n]}{I} = A \xrightarrow{\;\;\varphi\;\;} R.\]
                Dann gilt \(\varphi (x_i) = \widetilde{\varphi}(X_i)\) für alle \(i \in \{1, \ldots, n\}\) und \(\widetilde{\varphi}(f) = 0\) für jedes \(f \in I\), also folgt
                % Für jedes \(f \in I\) gilt weiterhin \(\widetilde{\varphi}(f) = 0\) und es folgt
                \[f(x) = f(\widetilde{\varphi}(X_1), \ldots, \widetilde{\varphi}(X_n)) = \widetilde{\varphi}(f) = 0,\]
                da \(\varphi\) \(k\)-Algebrahomomorphismus.
                Also ist \(x \in V_R(I)\) und \(ev_x = \varphi\).\qedhere
        \end{enumerate}
    \end{proof}

    \begin{beispiel}
        Betrachte für \(n \in \N\) den \(k\)-Funktor
        \[\mathbb{A}^n: k\textbf{-Alg} \to \textbf{Set}, \qquad R \mapsto R^n\]
        dieser ist wegen \(R^n \cong \Hom_{k\textbf{-Alg}}(\polynomials, R)\) ein affiner \(k\)-Funktor und wird repräsentiert vom Polynomring \(\polynomials\) in \(n\) Variablen.
    \end{beispiel}

    \begin{bemerkung}
        Aus dem Yoneda Lemma, bzw. \Cref{bem:2} folgt für \(\CC = k\textbf{-Alg}\), dass die Zuweisung \(A \mapsto h^A\) ein kontravarianter, volltreuer Funktor zwischen den \(k\)-Algebren und der Kategorie der \(k\)-Funktoren ist.
        Durch Einschränkung erhalten wir, dass die \emph{endlich erzeugten} \(k\)-Algebren dual zu den affinen \(k\)-Funktoren sind.
        Insbesondere verlieren wir keine Informationen beim Übergang der endlich erzeugten \(k\)-Algebra \(A\) zum affinen \(k\)-Funktor \(h^A\).
        Gleichzeitig haben wir unser Problem von früher gelöst, dass die \(k\)-Punkte \(V_k(I)\) für beliebigen Körper \(k\) nicht genug Informationen enthalten.
    \end{bemerkung}

    \begin{bemerkung}
        Tatsächlich haben wir sogar noch mehr gewonnen.
        Wir haben nicht nur die Kategorie der affinen \(k\)-Funktoren zur Verfügung sondern sogar die ganze Kategorie der \(k\)-Funktoren.
        Dies führt uns in die Richtung der Theorie der (affinen) Schemata.
    \end{bemerkung}
    

    % ****************************************
    \section{Algebraische Gruppen}
    % ****************************************

    \begin{definition}
        \begin{enumerate}
            \item Ein \emph{\(k\)-Gruppenfunktor} \(\mathcal{G}\) ist ein Funktor von der Kategorie der \(k\)-Algebren in die Kategorie \textbf{Grp} der Gruppen.
                Jedem \(k\)-Gruppenfunktor liegt ein \(k\)-Funktor \(\mathcal{G}^{\textbf{Set}}\) zugrunde durch die Komposition
                \[k\textbf{-Alg} \xrightarrow{\quad \mathcal{G} \quad} \textbf{Grp} \xrightarrow{\;\text{forget}\;} \textbf{Set}.\]

            \item Eine \emph{affine algebraische Gruppe} ist ein \(k\)-Gruppenfunktor \(\mathcal{G}\) sodass der zugrundeliegende \(k\)-Funktor \(\mathcal{G}^{\textbf{Set}} = \text{forget} \circ \mathcal{G}\) affin ist.
            
            \item Wenn \(\mathcal{G}\) eine affine algebraische Gruppe ist, dann wird \(\mathcal{G}^{\textbf{Set}}\) von einer eindeutigen endlich erzeugten \(k\)-Algebra \(A\) repräsentiert.
                Diese wird \emph{Koordinatenring}, oder \emph{Koordinatenalgebra} von \(\mathcal{G}\) genannt und wird mit \(k[\mathcal{G}]\) bezeichnet.

            \item Morphismen zwischen affinen algebraischen Gruppen \(\mathcal{G}\) und \(\mathcal{H}\) sind (wie bei \(k\)-Funktoren) natürliche Transformationen zwischen den \(k\)-Gruppenfunktoren.
        \end{enumerate}
    \end{definition}

    \begin{bemerkung}
        Die Eindeutigkeit des Koordinatenrings folgt wie bei \(k\)-Funktoren aus \Cref{bem:2} bzw. im wesentlichen aus \Cref{lem:3}, es arbeitet aber wieder einmal das Yoneda-Lemma im Hintergrund!
    \end{bemerkung}

    \begin{beispiel}
        \begin{enumerate}
            \item Der Funktor definiert durch
                \[k\textbf{-Alg} \longrightarrow \textbf{Grp}, \qquad R \mapsto (R, +)\]
                wird mit \(\mathbb{G}_a\) bezeichnet.
                Für jedes \(R \in \Ob(k\textbf{-Alg})\) können wir \(\mathbb{G}_a\) mit \(\Hom_{k\textbf{-Alg}}(k[X], R)\) identifizieren.
                Für den Koordinatenring \(k[\mathbb{G}_a]\) gilt dann \[k[\mathbb{G}_a] \cong k[X].\]
                Die affine algebraische Gruppe \(\mathbb{G}_a\) wird \emph{additive (algebraische) Gruppe über \(k\)} genannt.

            \item Für \(n \in \N\) betrachte \(\text{SL}_n\) als den Funktor
                \[k\textbf{-Alg} \longrightarrow \textbf{Grp}, \qquad R \mapsto \text{SL}_n(R).\]
                Damit wird \(\text{SL}_n\) zu einer affinen algebraischen Gruppe mit
                \[k[\text{SL}_n] \cong \ringquot{k[X_{11}, \ldots, X_{nn}]}{(\det (X_{ij}) - 1)}.\]

            \item Für \(n \in \N\) betrachte \(\text{GL}_n\) als den Funktor
                \[k\textbf{-Alg} \longrightarrow \textbf{Grp}, \qquad R \mapsto \text{GL}_n(R).\]
                Damit wird auch \(\text{GL}_n\) zu einer algebraischen Gruppe mithilfe des Tricks von Rabinowitsch
                \[k[\text{GL}_n] \cong \ringquot{k[X_1, \ldots, X_n, t]}{(t \cdot \det(X_{ij}) - 1)}.\]

            \item Der Funktor \(\text{GL}_1\) wird auch mit \(\mathbb{G}_m\) bezeichnet und entspricht dem Funktor
                \[k\textbf{-Alg} \longrightarrow \textbf{Grp}, \qquad R \mapsto (R^\times, \;\cdot\;).\]
                Für den Koordinatenring gilt dann
                \[k[\mathbb{G}_m] \cong \ringquot{k[X, t]}{(t \cdot X - 1)} \cong k[X, X^{-1}].\]
                Die algebraische Gruppe \(\mathbb{G}_m\) wird \emph{multiplikative (algebraische) Gruppe über \(k\)} genannt.

            \item Der Funktor \(\text{SL}_1\) widerum entspricht dem Funktor
                \[k\textbf{-Alg} \longrightarrow \textbf{Grp}, \qquad R \mapsto \{\ast\}\]
                und wird deshalb die \emph{triviale algebraische Gruppe \(\ast\) über \(k\)} genannt.
                Es gilt weiter
                \[k[\ast] \cong \ringquot{k[X]}{(X-1)} \cong k.\]

            \item Für \(n \in \N\) definiere den mit \(\mu_n\) bezeichneten Funktor
                \[k\textbf{-Alg} \longrightarrow \textbf{Grp}, \qquad R \mapsto \{r \in R \;\vert\; r^n = 1\}.\]
                \(\mu_n\) wird die \emph{algebraische Gruppe der \(n\)-ten Einheitswurzeln über \(k\)} genannt und es gilt
                \[k[\mu_n] \cong \ringquot{k[X]}{(X^n - 1)}.\]
        \end{enumerate}
    \end{beispiel}

    \begin{definition}
        Sei \(\mathcal{G}\) ein \(k\)-Gruppenfunktor.
        Für jedes \(R \in \Ob(k\textbf{-Alg})\) ist \(\mathcal{G}(R)\) also eine Gruppe.
        Die Multiplikation, die Inverse und das neutrale Element in \(\mathcal{G}(R)\) definieren natürliche Transformationen
        \begin{gather*}
            \mu : \mathcal{G} \times \mathcal{G} \longrightarrow \mathcal{G}, \\
            \iota : \mathcal{G} \longrightarrow \mathcal{G}, \\
            e : \ast \longrightarrow \mathcal{G}.
        \end{gather*}
        Nach \Cref{kor:1} induzieren diese (eindeutige) Co-Morphismen auf der \(k\)-Algebra \(k[\mathcal{G}]\)
        \begin{gather*}
            \Delta : k[\mathcal{G}] \longrightarrow k[\mathcal{G}] \otimes_k k[\mathcal{G}], \\
            S: k[\mathcal{G}] \longrightarrow k[\mathcal{G}], \\
            \varepsilon : k[\mathcal{G}] \longrightarrow k[1] = k.
        \end{gather*}
        Diese heißen \emph{Comultiplikation}, \emph{Coinverse} und \emph{Coeinheit}.
    \end{definition}

    \begin{beispiel}
        \begin{enumerate}
            \item Betrachte die additive algebraische Gruppe \(\mathcal{G} = \mathbb{G}_a : R \mapsto (R, +)\) mit darstellender Algebra \(k[\mathbb{G}_a] \cong k[X]\) dem Polynomring in einer Variablen.
                Dann ist für jede \(k\)-Algebra \(R \in \Ob(k\textbf{-Alg})\) folgende Abbildung eine Bijektion
                \[\beta: \Hom_{k\textbf{-Alg}}(k[X], R) \longrightarrow (R, +), \qquad \varphi \mapsto \varphi(X).\]
                Für \(\mathbb{G}_a \times \mathbb{G}_a\) gilt
                \[k[\mathbb{G}_a \times \mathbb{G}_a] \cong k[X] \otimes_k k[X] \cong k[X \otimes 1, 1 \otimes X]\]
                und analog ist folgende Abbildung eine Bijektion
                \[\alpha: \Hom_{k\textbf{-Alg}}(k[X \otimes 1, 1 \otimes X], R) \longrightarrow (R \times R, +), \qquad \varphi \mapsto (\varphi(X \otimes 1), \varphi(1 \otimes X)).\]
                Die natürliche Transformation \(\mu: \mathbb{G}_a \times \mathbb{G}_a \to \mathbb{G}_a\) induziert dann für jede \(k\)-Algebra \(R\) einen Morphismus
                \[\mu_R : \Hom_{k\textbf{-Alg}}(k[X \otimes 1, 1 \otimes X], R) \longrightarrow \Hom_{k\textbf{-Alg}}(k[X], R)\]
                und wir erhalten das kommutative Diagramm: 
                \begin{figure*}[ht!]
                    \centering
                    \begin{tikzcd}
                        {\Hom_{k\textbf{-Alg}}(k[X \otimes 1, 1 \otimes X], R)} \arrow[d, "\alpha"'] \arrow[r, "\mu_R"] & {\Hom_{k\textbf{-Alg}}(k[X], R)} \arrow[d, "\beta"] \\[2.25em]
                        {(R \times R, +)} \arrow[r, "\text{Addition}"]                       & {(R, +)}                                  
                    \end{tikzcd}
                \end{figure*} \\
                Insgesamt haben wir \[\mu_R(\varphi)(X) = \varphi(X \otimes 1) + \varphi(1 \otimes X) = \varphi(X \otimes 1 + 1 \otimes X)\]
                für jede \(k\)-Algebra \(R\).
                Nach \Cref{lem:yoneda} ist die Comultiplikation gegeben durch
                \[\Delta = \mu_{k[\mathcal{G}] \otimes k[\mathcal{G}]}(\id_{k[\mathcal{G}] \otimes k[\mathcal{G}]}) = \mu_{k[X] \otimes k[X]}(\id_{k[X] \otimes k[X]}),\]
                also \[\Delta(X) = X \otimes 1 + 1 \otimes X.\]
                Dadurch ist \(\Delta\) als \(k\)-Algebrahomomorphismus bereits festgelegt.
                Analog erhält man \[S(X) = -X \qquad\text{und}\qquad \varepsilon(X) = 0.\]

            \item Für die multiplikative Guppe \(\mathbb{G}_m : R \to (R^\times, \;\cdot\;)\), mit darstellender \(k\)-Algebra gegeben durch \(k[\mathbb{G}_m] = k[X, X^{-1}]\), ist die Multiplikation gegeben durch
                \[\mu_R(\varphi)(X) = \varphi(X \otimes 1)\varphi(1 \otimes X) = \varphi((X \otimes 1)(1 \otimes X)) = \varphi(X \otimes X),\]
                für jede \(k\)-Algebra \(R\).
                Die Comultiplikation ist dann gegeben durch 
                \[\Delta (X) = X \otimes X.\]
                Die Coinverse und Coeinheit sind gegeben durch
                \[S(X) = X^{-1} \qquad\text{und}\qquad \varepsilon(X) = 1.\]

            \item Für die affine algebraische Gruppe \(\text{GL}_n\) mit darstellender Algebra gegeben durch
                \[k[\text{GL}_n] = \ringquot{k[X_{ij}, d]}{(d \cdot \det(X_{ij}) - 1)}\]
                erhalten wir folgende Operationen
                \begin{gather*}
                    \Delta(X_{ij}) = \sum_{k = 1}^{n} X_{ik} \otimes X_{kj} \quad\text{bzw.}\quad \Delta(d) = d \otimes d, \\
                    S(X_{ij}) = d \cdot a_{ij} \quad\text{bzw.}\quad S(d) = \det(X_{ij}), \\
                    \varepsilon(X_{ij}) = \delta_{ij} \quad\text{bzw.}\quad \varepsilon(d) = 1.
                \end{gather*}
                dabei bezeichnen die \(a_{ij}\) die Einträge der Adjunkten\footnote[1]{\url{https://de.wikipedia.org/wiki/Adjunkte}}.
        \end{enumerate}
    \end{beispiel}

    \begin{lemma}\label{lem:1}
        Sei \(\mathcal{G}\) ein \(k\)-Funktor.
        Dann ist \(\mathcal{G} = \mathcal{H}^{\textbf{Set}}\) für einen \(k\)-Gruppenfunktor \(\mathcal{H}\) genau dann, wenn es natürliche Transformationen
        \begin{gather*}
            \mu: \mathcal{G} \times \mathcal{G} \longrightarrow \mathcal{G}, \\
            \iota: \mathcal{G} \longrightarrow \mathcal{G}, \\
            e: \ast \longrightarrow \mathcal{G}
        \end{gather*}
        gibt, sodass die folgenden Diagramme kommutieren:
        \begin{figure*}[ht!]
            \centering
            \begin{tikzcd}
                \mathcal{G} \times \mathcal{G} \times \mathcal{G} \arrow[rr, "\id\times \mu"] \arrow[d, "\mu\times \id"'] &  & \mathcal{G} \times \mathcal{G} \arrow[d, "\mu"] \\[2em]
                \mathcal{G} \times \mathcal{G} \arrow[rr, "\mu"]                                                &  & \mathcal{G}                          
            \end{tikzcd}
        \end{figure*}
        \begin{figure*}[ht!]
            \centering
            \begin{tikzcd}
                \mathcal{G} \times \ast \arrow[rr, "\id\times e"] \arrow[d, "\cong"'] &  & \mathcal{G} \times \mathcal{G} \arrow[d, "\mu"] \\[2em]
                \mathcal{G} \arrow[rr, Rightarrow, no head]                           &  & \mathcal{G}                          
            \end{tikzcd}\hspace*{4em}
            \begin{tikzcd}
                \ast\times \mathcal{G} \arrow[rr, "e \times\id"] \arrow[d, "\cong"'] &  & \mathcal{G} \times \mathcal{G} \arrow[d, "\mu"] \\[2em]
                \mathcal{G} \arrow[rr, Rightarrow, no head]                          &  & \mathcal{G}                          
            \end{tikzcd}
        \end{figure*}
        \begin{figure*}[ht!]
            \centering
            \begin{tikzcd}
                \mathcal{G} \arrow[rr, "{(\id, \iota)}"] \arrow[d] &  & \mathcal{G} \times \mathcal{G} \arrow[d, "\mu"] \\[2em]
                \ast \arrow[rr, "e"]                     &  & \mathcal{G}                          
            \end{tikzcd}\hspace*{4em}
            \begin{tikzcd}
                \mathcal{G} \arrow[rr, "{(\iota, \id)}"] \arrow[d] &  & \mathcal{G} \times \mathcal{G} \arrow[d, "\mu"] \\[2em]
                \ast \arrow[rr, "e"]                     &  & \mathcal{G}                          
            \end{tikzcd}
        \end{figure*}
    \end{lemma}

    \begin{proof}
        % Für jede \(k\)-Algebra \(R\) sind obige Diagramme ebenfalls kommutativ, da die \(\mathcal{G}\) kovariante Funktoren sind.
        % Dann stellt das erste Diagramm die Assoziativität in der Gruppe \(\mathcal{G}(R)\) dar, zweites und drittes die Existenz des neutralen Elements und die letzten Beiden die Existenz von Inversen Elementen.
        Kommt unser Funktor bereits von einem Gruppenfunktor, so ist die Aussage klar.
        Haben wir andererseits einen \(k\)-Funktor zusammen mit natürlichen Transformationen \(\mu, \iota\) und \(e\) sodass die Diagramme kommutieren, so definieren diese natürlichen Transformationen eine Gruppenstruktur auf den Mengen \(\mathcal{G}(R)\) für jede \(k\)-Algebra \(R\).
    \end{proof}

    \begin{bemerkung}
        Dies ist tatsächlich Relikt einer allgemeineren Konstruktion, nämlich Gruppenobjekte in allgemeinen Kategorien.
        Wir haben zum Beispiel schon die Gruppenobjekte in der Kategorie \textbf{Diff} der glatten Mannigfaltigkeiten, nämlich Liegruppen kennengelernt.
        In der Kategorie \textbf{Top} der topologischen Räume sind Gruppenobjekte die topologischen Gruppen.
        Insbesondere sind nach obigem Lemma algebraische Gruppen genau die Gruppenobjekte in der Kategorie der \(k\)-Algebren.
    \end{bemerkung}

    \begin{proposition}\label{prop:1}
        Sei \(A\) eine endlich erzeugte \(k\)-Algebra mit Multiplikation \(m: A \otimes A \to A\).
        Dann ist \(A\) die Koordinatenalgebra einer affinen algebraischen \(k\)-Gruppe genau dann, wenn es \(k\)-Algebrahomomorphismen
        \begin{gather*}
            \Delta: A \longrightarrow A \otimes A, \\
            S: A \longrightarrow A, \\
            \varepsilon: A \longrightarrow k
        \end{gather*}
        gibt, sodass die folgenden Diagramme kommutieren
        \begin{figure*}[ht!]
            \centering
            \begin{tikzcd}
                A \arrow[rr, "\Delta"] \arrow[d, "\Delta"'] &  & A\otimes A \arrow[d, "\Delta\otimes \id"] \\[2em]
                A \otimes A \arrow[rr, "\id\otimes \Delta"] &  & A\otimes A\otimes A                      
            \end{tikzcd}
        \end{figure*}
        \begin{figure*}[ht!]
            \centering
            \begin{tikzcd}
                A \arrow[rr, "\Delta"] \arrow[d, Rightarrow, no head] &  & A\otimes A \arrow[d, "\id\otimes \varepsilon"] \\[2em]
                A \arrow[rr, "\cong"]                                 &  & A \otimes k                                
            \end{tikzcd}\hspace*{4em}
            \begin{tikzcd}
                A \arrow[rr, "\Delta"] \arrow[d, Rightarrow, no head] &  & A\otimes A \arrow[d, "\varepsilon\otimes\id"] \\[2em]
                A \arrow[rr, "\cong"]                                 &  & k \otimes A                                
            \end{tikzcd}
        \end{figure*}
        \begin{figure*}[ht!]
            \centering
            \begin{tikzcd}
                A \arrow[rr, "\varepsilon"] \arrow[d, "\Delta"']   &  & k \arrow[d, "\eta"] \\[2em]
                A \otimes A \arrow[rr, "m \circ(\id\otimes S)"] &  & A                  
            \end{tikzcd}\hspace*{4em}
            \begin{tikzcd}
                A \arrow[rr, "\varepsilon"] \arrow[d, "\Delta"']   &  & k \arrow[d, "\eta"] \\[2em]
                A \otimes A \arrow[rr, "m \circ(S \otimes \id)"] &  & A                  
            \end{tikzcd}
        \end{figure*}
    \end{proposition}
    \begin{proof}
        Da nach \Cref{def:2} die dualen Morphismen auf den Koordinatenringen eindeutig sind (Yoneda im Hintergrund!) folgt diese Proposition bereits aus \Cref{lem:1} durch dualisieren.
        % Wir müssen lediglich die duale Abbildung zu \[(\id, \iota): \mathcal{G} \to \mathcal{G} \times \mathcal{G}\] bestimmen.
        Wir müssen uns lediglich davon überzeugen, dass die zur Abbildung \[(\id, \iota): \mathcal{G} \to \mathcal{G} \times \mathcal{G}\]
        duale Abbildung durch \(m \circ (\id \otimes S)\) gegeben ist.
        % Es muss nur noch die Abbildung \[(\id, \iota): G \to G \times G\] dualisiert werden.
        Betrachte dazu die Komposition
        \[\mathcal{G} \xrightarrow{\;\text{diag}\;} G \times \mathcal{G} \xrightarrow{\;\id \times \iota\;} \mathcal{G} \times \mathcal{G},\]
        mit \(\text{diag}_R: \mathcal{G}(R) \to \mathcal{G}(R) \times \mathcal{G}(R), \; g \mapsto (g, g)\).
        Es ist also zu zeigen, dass die duale Abbildung von \(\text{diag}: \mathcal{G} \to \mathcal{G} \times \mathcal{G}\) genau die Abbildung \(m: A \otimes A \to A\) ist.
        Dies folgt aus \Cref{lem:yoneda}, da 
        \[\text{diag}_A(\id_A): A \otimes A \longrightarrow A, \; a \otimes b \mapsto \id_A(a)\id_A(b) = ab = m(a \otimes b).\qedhere\]
    \end{proof}

    \begin{bemerkung}
        Es ist bemerkenswert wie viel das Yoneda-Lemma im Hintergrund arbeitet und wie viel Theorie wir durch geschickte Definitionen herauskristallisieren können.
    \end{bemerkung}

    
    % ****************************************
    \section{Hopf Algebren und Algebraische Gruppen}
    % ****************************************

    \begin{definition}\label{def:1}
        \begin{enumerate}
            \item Eine \(k\)-Algebra \(A\) mit \(k\)-Algebrahomomorphismen \(\Delta, \varepsilon, S\), sodass die Diagramme aus \Cref{prop:1} kommutieren heißt \emph{(kommutative) Hopf Algebra}.
                Drücken wir die Diagramme als Formeln aus, erhalten wir die Axiome
                \begin{gather*}
                    (\id \otimes \Delta) \circ \Delta = (\Delta \circ \id) \circ \Delta, \\
                    m \circ (\id \circ \varepsilon) \circ \Delta = \id = m \circ (\varepsilon \otimes \id) \circ \Delta, \\
                    m \circ (\id \otimes S) \circ \Delta = \eta \circ \varepsilon = m \circ (S \otimes \id) \circ \Delta.
                \end{gather*}

            \item Seien \(A, B\) zwei Hopf-Algebren.
                Ein Hopf-Algebrahomomorphismus \(f: A \to B\) ist ein \(k\)-Algebrahomomorphismus der mit den Morphismen \(\Delta, \varepsilon\) und \(S\) verträglich ist.
                Also
                \begin{gather*}
                    \Delta_B \circ f = (f \otimes f) \circ \Delta_A, \\
                    S_B \circ f = f \circ S_A, \\
                    \epsilon_B \circ f = \epsilon_A.
                \end{gather*}
        \end{enumerate}
    \end{definition}

    \begin{korollar}
        Die Kategorie der affinen algebraischen \(k\)-Gruppen ist dual (oder antivalent) zur Kategorie der kommutativen endlich erzeugten Hopf-Algebren.
    \end{korollar}
    \begin{proof}
        Dies ist eine direkte Kosequenz von \Cref{prop:1}.
    \end{proof}

    \begin{bemerkung}
        Es ist also gleichwertig, einen Funktor \(\mathcal{G}: k\textbf{-Alg} \to \textbf{Grp}\) anzugeben, sodass die Komposition \(\text{forget} \circ \mathcal{G}\) darstellbar ist, oder ein Paar \((A, \Delta)\) anzugeben, bestehend aus einer \(k\)-Algebra \(A\) und einem Morphismus \(\Delta: A \to A \otimes A\) der \(h^A(R)\) für jede \(k\)-Algebra \(R\) zu einer Gruppe macht.
    \end{bemerkung}
    % \begin{proof}
    %     \begin{enumerate}
    %         \item[`\(\Rightarrow\)'] Setze \(A = \Hom_{\CC^\vee}(\mathbb{A}^1, \mathcal{G}^{\text{Set}})\) zusammen mit dem Morphismus \(A \to A \otimes A\) der (nach Yoneda) zur natürlichen Transformation \(\mathcal{G} \times \mathcal{G} \to \mathcal{G}\) gehört.
    %         \item[`\(\Leftarrow\)'] Setze \(\mathcal{G} = h^A\) zusammen mit der Multiplikation \(h^\Delta\).
    %     \end{enumerate}
    % \end{proof}


    % % ****************************************
    % \section{Affine algebraische Gruppen sind linear}
    % % ****************************************

    % \begin{definition}
    %     Sei \(\mathcal{G}\) eine affine algebraische Gruppe über \(k\), sowie \(V\) ein \(k\)-Vektorraum.
    %     Eine \emph{Darstellung} von \(\mathcal{G}\) ist eine natürliche Transformation
    %     \[\rho : \mathcal{G} \longrightarrow \text{GL}_V,\]
    %     wobei \(\text{GL}_V\) den \(k\)-Gruppenfunktor
    %     \[\text{GL}_V(R) := \text{GL}(R \otimes_k V)\]
    %     bezeichnet.
    %     Dabei ist \(R \otimes_k V\) ein freier \(R\)-Modul und \(\text{GL}(R \otimes_k V)\) bezeichnet die Gruppe der Automorphismen dieses \(R\)-Moduls.
    % \end{definition}

    % \begin{definition}
    %     Sei \(A\) eine Hopf-Algebra über \(k\).
    %     Ein \emph{\(A\)-Comodul} ist ein Paar \((V, m)\), wobei \(V\) ein \(k\)-Vektorraum und \(m: V \to A \otimes_k V\) eine \(k\)-lineare Abbildung ist, sodass gilt %folgende Axiome gelten
    %     \begin{gather*}
    %         (\id_A \otimes m) \circ m = (\Delta \otimes \id_V) \circ m, \\
    %         (\varepsilon \otimes \id_V) \circ m = \id_V.
    %     \end{gather*}
    %     Als Diagramme ausgedrückt heißt dies, dass die folgenden beiden Diagramme kommutieren
    %     \begin{figure*}[ht!]
    %         \centering
    %         \begin{tikzcd}
    %             V \arrow[rr, "m"] \arrow[d, "m"']            &  & A \otimes V \arrow[d, "\id \otimes m"] \\[2em]
    %             A \otimes V \arrow[rr, "\Delta \otimes \id"] &  & A \otimes A \otimes V                 
    %         \end{tikzcd}\hspace*{4em}
    %         \begin{tikzcd}
    %             V \arrow[rr, "m"] \arrow[d, Rightarrow, no head] &  & A \otimes V \arrow[d, "\varepsilon \otimes \id"] \\[2em]
    %             V \arrow[rr, "\cong"]                            &  & k\otimes V                                   
    %         \end{tikzcd}
    %     \end{figure*}
    % \end{definition}

    % \begin{proposition}\label{prop:2}
    %     Sei \(\mathcal{G}\) eine affine algebraische \(k\)-Gruppe mit Koordinatenring \(A = k[\mathcal{G}]\).
    %     \begin{enumerate}
    %         \item Sei \(\rho : \mathcal{G} \to \text{GL}_V\) eine Darstellung und sei \(m\) die Einschränkung von
    %             \[\rho_A(\id_A) \in \text{GL}_V(A) = \text{GL}(A \otimes_k V)\]
    %             auf \(V\).
    %             Dann ist das Paar \((V, m)\) ein \(A\)-Comodul.

    %         \item Umgekehrt, sei \((V, m)\) ein \(A\)-Comodul, und sei \(\rho: \mathcal{G} \to \text{GL}_V\) die \emph{natürliche Darstellung} gegeben durch
    %             \[\rho_R(g) := (g \otimes \id_V) \circ m \qquad \text{für alle}\; g \in \mathcal{G}(R) = \Hom_{k\textbf{-Alg}}(A, R).\]
    %             Dann ist \(\rho\) eine Darstellung für \(\mathcal{G}\), genannt \(\mathcal{G}\)-Darstellung.
    %     \end{enumerate}
    % \end{proposition}
    % \begin{proof}[Ohne Beweis.]
    % \end{proof}
    
    % \begin{bemerkung}\label{bem:3}
    %     Sei \(\mathcal{G}\) eine affine algebraische Gruppe mit Koordinatenalgebra \(k[\mathcal{G}]\) und \(\Delta\) die auf \(k[\mathcal{G}]\) definierte Comultiplikation.
    %     Dann ist das Paar \((k[\mathcal{G}], \Delta)\) ein \(k[\mathcal{G}]\)-Comodul und nach \Cref{prop:2} induziert dieser eine Darstellung von \(\mathcal{G}\) auf sich selbst.
    % \end{bemerkung}

    % \begin{definition}
    %     Die induzierte Darstellung aus \Cref{bem:3} wird \emph{reguläre Darstellung} genannt.
    % \end{definition}

    % \begin{lemma}\label{lem:2}
    %     Jede \(\mathcal{G}\)-Darstellung \((V, m)\) einer affinen algebraischen Gruppe \(\mathcal{G}\) ist lokal endlich.
    % \end{lemma}
    % \begin{proof}[Ohne Beweis]
    % \end{proof}

    % \begin{satz}
    %     Sei \(\mathcal{G}\) eine affine algebraische Gruppe über \(k\).
    %     Dann gibt es einen endlich-dimensionalen \(k\)-Vektorraum \(V\) und einen injektiven Morphismus \(\rho: \mathcal{G} \hookrightarrow \text{GL}_V\).
    % \end{satz}
    % \begin{proof}
    %     Sei \(A = k[\mathcal{G}]\) die endlich erzeugte Koordinatenalgebra von \(\mathcal{G}\) und \(W\) ein endlichdimensionaler \(k\)-Untervektorraum von \(A\), der \(A\) als \(k\)-Algebra erzeugt.
    %     Nach \Cref{lem:2} ist \(W\) in einer endlichdimensionalen Unterdarstellung \(V\) der regulären Darstellung \((A, \Delta)\) enthalten.
    %     Wir bezeichnen mit 
    %     \[\rho: G \longrightarrow \text{GL}_V\]
    %     die zugehörige natürliche Transformation.
    %     Zu zeigen ist, dass \(\rho\) injektiv ist.
    %     Man kann zeigen, dass dies äquivalent dazu ist, dass der duale Morphismus
    %     \[\rho^\ast: k[\text{GL}_V] \longrightarrow A\]
    %     surjektiv ist.
    %     Sei \(\{v_1, \ldots, v_n\}\) eine Basis von \(V\), und betrachte
    %     \[\Delta(v_j) = \sum_{i=1}^{n} f_{ij} \otimes v_i\]
    %     mit \(f_{ij} \in A\).
    %     Wegen \Cref{prop:2} ist die natürliche Transformation \(\rho\) gegeben durch
    %     \[\rho_R(g)(v_j) = (g \otimes \id_V)(\Delta(v_j)) = \sum_{i=1}^{n} g(f_{ij}) \otimes v_i\]
    %     für alle \(g \in \mathcal{G}(R) \cong \Hom_{k\textbf{-Alg}}(A, R)\) und alle \(j \in \{1, \ldots, n\}\), also wird \(\rho_R(g)\) von der Matrix
    %     \[\rho_R(g) = (g(f_{ij}))_{ij} \in \text{GL}_V(R)\]
    %     dargestellt.
    %     Es folgt weiter
    %     \[\rho^\ast(X_{ij}) = \rho_A(\id_A)(X_{ij}) = f_{ij}\]
    %     für alle \(i,j\), wobei \(X_{ij}\) die Koordinatenfunktionen von 
    %     \[k[\text{GL}_V] \cong \ringquot{k[X_{11}, \ldots, X_{nn}, d]}{(d\cdot \det(X_{ij}) - 1)}.\]
    %     Andererseits folgt aus \Cref{def:1}, dass
    %     \[v_j = m(\id \otimes \varepsilon)\Delta(v_j) = m \left( \sum_{i=1}^{n} f_{ij} \otimes \varepsilon(v_i) \right) = \sum_{i=1}^{n} \varepsilon(v_i) \cdot f_{ij}.\]
    %     Es folgt \(v_j \in \text{im} \; \rho^\ast\), für alle \(j \in \{1, \ldots, n\}\).
    %     Da \(A\) von den Elementen \(v_1, \ldots, v_n\) als \(k\)-Algebra erzeugt wird, folgt dass \(\rho^\ast\) surjektiv ist, was zu zeigen war.
    % \end{proof}

    % \begin{korollar}
    %     Affine algebraische Gruppen über \(k\) sind linear, d.h. sie sind abgeschlossene Untergruppen einer Gruppe \(\text{GL}_n\).
    %     Insbesondere sind alle affine algebraische Gruppen Matrixgruppen.
    % \end{korollar}

    % ****************************************
    \section{Literatur}
    % ****************************************

    Es gibt auf der Homepage von J. S. Milne \url{https://www.jmilne.org/math/} einige gute Skripte, aber auch Bücher frei Verfügbar.
    Unter anderem auch über Algebraische Gruppen.
    
    \begin{enumerate}
        \item J. S. Milne `Algebraic groups' v. 2017: \url{www.jmilne.org/math/Books/iAG2017.pdf}
        \item J. S. Milne `Algebraic groups' v. 2022: \url{www.jmilne.org/math/Books/iAG2022.pdf}
        \item W. C. Waterhouse `Introduction to affine group schemes'
    \end{enumerate}
\end{document}